\let\RM\rmfamily
\let\TT\ttfamily
\let\SF\sffamily
\let\SC\scshape
\let\IT\itshape
\let\BF\bfseries


\usepackage{amsmath, amsthm, amssymb}   % math
\usepackage{mathtools}                  % mathclap
    \newtheorem{theorem}{Theorem}
\usepackage{graphicx}       % graphics
\usepackage[table]{xcolor}  % graphics
    \definecolor{B}    {HTML}{2b66d3}
    \definecolor{B2}   {HTML}{003399}
    \definecolor{R}    {HTML}{c9171e}
    \definecolor{R2}   {HTML}{d7003a}
    \definecolor{INK}  {HTML}{595857}
    \definecolor{Y}    {HTML}{f1c40f}
    \definecolor{G}    {HTML}{009a00}
    \definecolor{GRAY} {HTML}{808080}
    \definecolor{MAUVE}{HTML}{9400D1}
\usepackage{url}                  			% hyperlink
\usepackage[pdfencoding=auto,psdextra]{hyperref}
    \hypersetup{pdfborder={0 0 0}}
\usepackage{adjustbox}                      % box
\usepackage{booktabs, multirow}             % table
\usepackage{caption, subcaption}            % caption
    \captionsetup[figure]   {font=small}
    \captionsetup[table]    {font=small}
    \captionsetup[subfigure]{font=footnotesize}
    \captionsetup[subtable] {font=footnotesize}
    \captionsetup[algorithm]{font=footnotesize}
    \captionsetup{width=.8\linewidth, justification = raggedright}
\usepackage{multicol}
\usepackage{soul}
\usepackage{enumitem}                       % enhanced items
    \setlist{noitemsep, topsep=0.25ex}
\usepackage{lipsum}                         % pseudo texts
\usepackage{setspace}                       % spacing
\usepackage{algorithm, algpseudocode}       % algorithm
    \algrenewcommand{\alglinenumber}[1]{{\scriptsize\BF\TT\color{R}#1}}
\usepackage{footnote}                       % enhanced footnote
\usepackage[final]{listings}                % code listing; show in draft mode
    \lstset{language=bash}
    % string styles
    \lstset{
        basicstyle=\TT\small,
        stringstyle=\color{R},
        breaklines=true,
        keywordstyle=\color{B}\BF,
        commentstyle=\color{gray},
        showspaces=false,
        showstringspaces=false,
        numberstyle=\BF\TT\scriptsize\color{R},
        stepnumber=1, % the step between two line-numbers. If it's 1, each line will be numbered
        tabsize=2,
        numbers=none, % {none, left, right}; position of line number
        numbersep=10pt, % how far the line-numbers are from the code
        escapeinside={Ω}{≈},
%       stepnumber=5, firstnumber=1, numberfirstline=false,
        stepnumber=1, firstnumber=1, numberfirstline=true,
    }
    % string position
    \lstset{
        backgroundcolor=\color{black!2}, rulecolor=\color{black!25}
    }
    \lstset{
        frame=tb,               % t for top, single-line
        % framesep=0pt,
        xleftmargin=6pt,        % text move away from left border by 6pt
        framexleftmargin=6pt,   % frame move away from (right-hand) text by 6pt
        framextopmargin=1pt,    %
        framexbottommargin=1pt, %
    }

\usepackage{tikz}
    \newcommand*\Circled[1]{% require `tikz`
        \tikz[baseline=(char.base)]{%
            \node[shape=circle, fill=black, inner sep=0.6pt] (char) {%
                \textcolor{white}{\SF\BF#1}}; }}

\usepackage{transparent}

%%% macro
\setlength{\parindent}{0pt}
\setlength{\parskip}{6pt plus 2pt minus 1pt}
\providecommand{\tightlist}{%
  \setlength{\itemsep}{0pt}\setlength{\parskip}{0pt}}
  
\newcommand{\passthrough}[1]{\lstset{mathescape=false}#1\lstset{mathescape=true}}
